

%%%%
%%%%%%%%%%%%%%% GdB %%%%%%%%%%%%%%%
%%%%


\newcommand*{\ExactPlotGdb}[1]
{
    % Exact plot of GdB
    \edef\GdbReelSum{}
    \pgfplotsset{GdbReelSum/.list={1,...,\WcArraylen}}
    \expanded{\noexpand
      \addplot [samples at={\mySamplLst}] %=200,domain=\Wmin:\Wmax,mark=+]
      {   GdB(\GainTot)
        + GdBInteg(x)*\NbItegr
          \GdbReelSum
      }[LieuReel,#1];
    }
}

\newcommand*{\AsymptoticPlotGdb}[2]
{
  % Asymptotic plot
    % Create a list of domains (choice here to overlap)
      \xdef\alist{(\Wmin):}
      \foreachitem\i\in\WcArray[]{
        \xdef\alist{\alist(#1*\i),(\i/#1):}
      }
      \xdef\alist{\alist(\Wmax)}
      \readlist\alistlst{\alist}
    % Plot each line
      \foreach \i [count=\c] in \alist {
        \edef\GdbAsymSum{}
        \ifnum\c >1  
          \pgfmathparse{\c-1}
          \pgfplotsset{GdbAsymSum/.list={1,...,\pgfmathresult}}
        \fi
        \expanded{\noexpand%
          \addplot [domain=\i,samples=2]
          {   GdB(\GainTot)
            + GdBInteg(x)*\NbItegr
             \GdbAsymSum
          }[Asymp,#2];
        }
      }
}

\newcommand*{\AsymptoticDecompGdb}[1]
{
  \foreach \i [count=\c] in {1,...,\WcArraylen} {
    \edef\GdbAsymSum{}
    \pgfplotsset{GdbAsymSum/.list={1,...,\c}}
    \pgfmathparse{int(abs(\OrdrArray[\c]))}
    \ifnum\pgfmathresult=2
      \ifdim\ZArray[\c]pt >1pt
        \pgfmathsetmacro\Ws{\WcArray[\c]*(\ZArray[\c]+sqrt(\ZArray[\c]^2-1))}
        \pgfmathsetmacro\Wi{\WcArray[\c]*(\ZArray[\c]-sqrt(\ZArray[\c]^2-1))}
        \addplot[domain=\Wi:\Ws,samples=2]
        {   
            (x < (\Ws/2)) * ((\OrdrArray[\c])*20*log10(x/\WcArray[\c]))
            + GdB(\GainTot)
            + GdBInteg(x)*\NbItegr
            \GdbAsymSum
        }[Asymp,densely dotted,#1];
     \fi
   \fi
  }
}

\newcommand\ycoord[2][center]{{%
   % The actual point of interest
   \pgfpointanchor{#2}{#1}%
   \pgfgetlastxy{\ix}{\iy}%
   % (0,0)
   \pgfplotspointaxisxy{0}{0}%
   \pgfgetlastxy{\ox}{\oy}%
   % (1,1)
   \pgfplotspointaxisxy{1}{1}%
   \pgfgetlastxy{\ux}{\uy}%
   \pgfmathparse{(\iy-\oy)/(\uy-\oy)}%
   \pgfmathprintnumber[fixed, precision=2]{\pgfmathresult}%
   \expandafter\xdef\csname ycoord#2\endcsname{\pgfmathresult}%
}}

\newcommand\xcoord[2][center]{{%
   % The actual point of interest
   \pgfpointanchor{#2}{#1}%
   \pgfgetlastxy{\ix}{\iy}%
   % (0,0)
   \pgfplotspointaxisxy{1}{0}%
   \pgfgetlastxy{\ox}{\oy}%
   % (1,1)
   \pgfplotspointaxisxy{10}{0}%
   \pgfgetlastxy{\ux}{\uy}%
   \pgfmathparseFPU{10^((\ix-\ox)/(\ux-\ox))}%
   \pgfmathprintnumber[fixed, precision=10]{\pgfmathresult}%
   \expandafter\xdef\csname xcoord#2\endcsname{\pgfmathresult}%
}}

\newcommand*{\InfoResonance}[1]
{
  \foreach \i [count=\c] in {1,...,\WcArraylen} {
    \edef\GdbReelSum{}
    \pgfplotsset{GdbReelSum/.list={1,...,\WcArraylen}}
    \pgfmathparse{int(abs(\OrdrArray[\c]))}
    \ifnum\pgfmathresult=2
      \ifdim\ZArray[\c]pt <.707pt
        \pgfmathsetmacro\Wr{\WcArray[\c]*sqrt(1-2*(\ZArray[\c]^2))} 
        \addplot [mark=none,samples at={\Wr}] {
            + GdB(\GainTot)
            + GdBInteg(x)*\NbItegr
            \GdbReelSum
        } node (PtMax) {};
        %https://tex.stackexchange.com/questions/22018/how-to-access-xmin-xmax-ymin-ymax-from-within-pgfplots-axis-environment
        %\pgfmathparseFPU{\Wr}
        \draw[Asymp,densely dotted,#1] (PtMax.center-|\Wmin,0) --(PtMax.center)  -- (PtMax.center|-current axis.below origin) ;
        % \draw[Asymp,densely dotted,#1] (PtMax.center-|\Wmin,0) --(PtMax.center)node[midway,anchor=south,inner sep=0] {\ycoord{PtMax}}  -- (PtMax.center|-current axis.below origin) node[midway,anchor=west,inner sep=0] {\xcoord{PtMax}} ;
        %\node at (current axis.below origin) {\xcoord{PtMax}};
        % \pgfmathsetmacro\PWr{\pgfmathprintnumber[fixed, precision=4]{\pgfmathresult}}
     \fi
  \fi
  }
}


%%%%
%%%%%%%%%%%%%%% Phase %%%%%%%%%%%%%%%
%%%%

\newcommand*{\ExactPlotPhase}[1]
{
  % Exact plot of phy
    \edef\PhyReelSum{}
    \pgfplotsset{PhyReelSum/.list={1,...,\WcArraylen}}
    \expanded{\noexpand
      \addplot [samples at={\mySamplLst}]
      {
        +\IntegrPhy*\NbItegr
        \PhyReelSum
      }[LieuReel,#1];
    }
}

\newcommand*{\AsymptoticPlotPhase}[1]
{
  % Asymptotic plot
    % Create a list of points to plot the steps
      \xdef\alist{(\Wmin,\IntegrPhy*\NbItegr) }
      \foreach \i in {1,...,\WcArraylen} {
        \xdef\alist{\alist(\WcArray[\i],\IntegrPhy*\NbItegr}
        \ifnum\i>1  
          \pgfmathparse{\i-1}
          \foreach \k in {1,...,\pgfmathresult} {
            \xdef\alist{\alist\Saut{\OrdrArray[\k]}}
          }
        \fi
        \xdef\alist{\alist)  (\WcArray[\i],\IntegrPhy*\NbItegr}
        \foreach \k in {1,...,\i} {
          \xdef\alist{\alist\Saut{\OrdrArray[\k]}}
        }
        \xdef\alist{\alist)}
      }
      \xdef\alist{\alist(\Wmax,\IntegrPhy*\NbItegr}
      \foreach \k in {1,...,\WcArraylen} {
        \xdef\alist{\alist\Saut{\OrdrArray[\k]}}
      }
      \xdef\alist{\alist)}
    % Plot
	  \addplot[mark=none] coordinates %[mark=none]
	  { \alist }[Asymp];
}


\newcommand*{\AsymptoticDecompPhase}[1]
{
\xdef\alist{}
  \foreach \i [count=\c] in {1,...,\WcArraylen} {
    \pgfmathparse{int(abs(\OrdrArray[\c]))}
    \ifnum\pgfmathresult=2
      \ifdim\ZArray[\c]pt >1pt
        \pgfmathsetmacro\Ws{\WcArray[\c]*(\ZArray[\c]+sqrt(\ZArray[\c]^2-1))}
        \pgfmathsetmacro\Wi{\WcArray[\c]*(\ZArray[\c]-sqrt(\ZArray[\c]^2-1))}
        %%%
          \xdef\alist{\alist(\Wi,\IntegrPhy*\NbItegr}
          \xdef\blist{}
          \ifnum\i>1  
            \pgfmathparse{\i-1}
            \foreach \k in {1,...,\pgfmathresult} {
              \xdef\blist{\blist\Saut{\OrdrArray[\k]}}
            }
          \fi
          \xdef\alist{\alist\blist) (\Wi,\IntegrPhy*\NbItegr}
          \xdef\alist{\alist\blist\Saut{\OrdrArray[\i]}*.5}
          \xdef\alist{\alist) (\Ws,\IntegrPhy*\NbItegr}
          \xdef\alist{\alist\blist\Saut{\OrdrArray[\i]}*.5}
          \xdef\alist{\alist) (\Ws,\IntegrPhy*\NbItegr}
          \foreach \k in {1,...,\i} {
            \xdef\alist{\alist\Saut{\OrdrArray[\k]}}
          }
          \xdef\alist{\alist)}
        %%%
	  \addplot[mark=none] coordinates %[mark=none]
        { \alist }[Asymp,densely dotted,#1];%,#4];
     \fi
   \fi
  }
}


% Display the maximum : 
    % https://tex.stackexchange.com/questions/46176/pgfplots-mark-max-min-value-of-a-function
% Find intersection betwwen functions
    % https://tex.stackexchange.com/questions/303142/value-variable-in-an-addplot-domain